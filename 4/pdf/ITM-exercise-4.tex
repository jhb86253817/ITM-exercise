\documentclass{article}
\usepackage{amsmath}
\usepackage[pdftex]{graphicx}
\usepackage{subfigure}
\usepackage{indentfirst}

\title{Information Theoretic Modeling -- Exercise 4}
%\author{Haibo Jin}
\date{}


\bibliographystyle{plain}

\begin{document}

\maketitle

{\centering \large \textbf{Haibo Jin}}

{\centering \large \textbf{Student number: 014343698}}

\section{Problem 1}

\subsection*{(a)}

The probabilities of symbols $\{a,b,c,!\}$ is given as $p(a)$=0.05, $p(b)$=0.5, $p(c)$=0.35, $p(!)$=0.1, we can create a table illustrating the first iteration of arithmetic coding, which is also known as Shannon-Fano-Elias coding. Please see Table 1 for more details.
 
\begin{table}[h]
\centering
\begin{tabular}{|c | c| c| c|}
\hline
$x$	&$p(x)$	&$F(x)$	&$I(x)$\\
\hline
$a$	&0.05  &0.05  &[0, 0.05)\\
\hline
$b$	&0.5  &0.55  &[0.05, 0.55)\\
\hline
$c$	&0.35  &0.9  &[0.55, 0.9)\\
\hline
$!$	&0.1  &1  &[0.9, 1)\\
\hline
\end{tabular}
\caption{First iteration of arithmetic coding, $p(x)$ is symbol distribution, $F(x)$ is cumulative function and $I(x)$ is the interval for symbol $x$}
\end{table}

The target message is $cab!$, so we choose $I(c)$ from the first iteration and will be further split in the next iteration. So now we have $I(c)=[0.55, 0.9)$.

\vspace{3mm}

The second symbol is $a$, we can use similar method to allocate an interval for $a$. But the total range of the interval is no longer $[0, 1]$ but $I(c)$. Then the probability of $a$ is $(0.9-0.55)\times0.05=0.0175$. So $I(ca)=[0.55, 0.5675)$.

\vspace{3mm}

The third symbol is $b$. We know its probability is between $(0.5675-0.55)\times0.05=0.000875$ and $(0.5675-0.55)\times0.55=0.009625$. So $I(cab)=[0.550875, 0.559625)$.

\vspace{3mm}

The last symbol is $!$ and we can get its proportion of probability is $(0.559625-0.550875)\times0.9=0.007875$. Finally, the interval $I(cab!)$ will be $\textbf{[0.55875, 0.559625)}$. 

\subsection*{(b)}

Based on the previous calculation, $I(cab!)=[0.55875, 0.559625)$

\subsubsection*{i.}

The shortest codeword within the interval is $\textbf{0.559}$.

\subsubsection*{ii.}

The codewords that satisfy the condition are 

\vspace{1mm}

$\textbf{0.5588$, $0.5589$, $0.5590$, $0.5591$, $0.5592$, $0.5593$, $0.5594$, $0.5595}$.

\subsection*{(c)}

Since the symbol probabilities are slightly changed, we need to calculate the interval of $cab!$ again and the method is basically the same as previous.

\vspace{2mm}

The new symbol probabilities are $p(a)=2/32$, $p(b)=16/32$, $p(c)=11/32$, $p(!)=3/32$.

\vspace{2mm}

We can get $I(c)=[18/32, 29/32)$, $I(ca)=[288/512, 299/512)$, $I(cab)=[4619/8192, 4707/8192)$, $I(cab!)=[18795/32768, 4707/8192)$.

\vspace{2mm}

By transforming the real numbers to binaries, $I(cab!)$ can be rephrased as $[0.100100101101011, 0.1001001100011)$.

\vspace{2mm}

Now we can answer that shortest codeword within the interval is $\textbf{0.10010011}$.

\vspace{2mm}

The shortest codeword that all its continuations are also within the interval is $\textbf{0.10010010111}$.

\section{Problem 2}

$D=0011$

\vspace{3mm}

The two-part code-length can be represented as $L=l_1(\theta)+log_2\frac{1}{p_\theta(D)}$. 

\vspace{3mm}

For $\theta=0.25$, $p_{0.25}(0011)=0.25^2\times0.75^2=0.03515625$

$l(0.25)=2$

$L=2+4.83=6.83$

\vspace{3mm}

For $\theta=0.5$, $p_{0.5}(0011)=0.5^2\times0.5^2=0.0625$

$l(0.25)=1$

$L=1+4=5$

\vspace{3mm}

For $\theta=0.75$, $p_{0.25}(0011)=0.75^2\times0.25^2=0.03515625$

$l(0.25)=2$

$L=2+4.83=6.83$

\vspace{3mm}

So, the expected two-part code-length of sequence $0011$ is $E(L)=(6.83\times2+5)/3=\textbf{6.22}$

\section{Problem 3}

Since the parameter prior is uniform, so $\omega(\theta)=1$, then we can get

\vspace{2mm}

$p(D)=\int_{\Theta}p_{\theta}(D)\omega(\theta)d\theta=\int_{\Theta}p_{\theta}(D)d_{\theta}=\int_{\Theta}\theta^k(1-\theta)^{n-k}d_\theta$

\vspace{2mm}

Use Beta-Binomial distribution, $p(D)$ can be calculated easily since we can control the value of $\theta$ by controlling the parameter $\alpha$ and $\beta$ in Beta-Binomial distribution. In our case, the target sequence is $0011$, which is just a discrete point in Beta-Binomial distribution with $\alpha=1$, $\beta=1$, $n=4$ and $k=2$. So the probability of this point is

\vspace{3mm}

$pmf=\frac{B(k+\alpha, n-k+\beta)}{B(\alpha, \beta)}=\frac{B(2+1,4-2+1)}{B(1,1)}=\frac{B(3,3)}{B(1,1)}=0.0333$

\vspace{3mm}

So the mixture code-length of $0011$ is $-log_2p(D)=\textbf{4.9069}$.

\section{Problem 4}

\subsection*{(a)}

$C={4\choose0}\times1+{4\choose1}\times\frac{1}{4}\times\frac{3}{4}^3+{4\choose2}\times\frac{2}{4}\times\frac{2}{4}^2+{4\choose3}\times\frac{3}{4}^3\times\frac{1}{4}+{4\choose4}\times1$

\vspace{2mm}

$=1+\frac{27}{64}+\frac{3}{8}+\frac{27}{64}+1$

\vspace{2mm}

$=\frac{103}{32}=\textbf{3.21875}$

\subsection*{(b)}

The maximum likelihood model is $p_{\hat{\theta}}(0011)=\frac{1}{2}^2\times\frac{1}{2}^2=\frac{1}{16}$

\vspace{2mm}

So the normalized maximum likelihood model is $p_{nml}(0011)=\frac{p_{\hat{\theta}}(0011)}{C}=\frac{1}{16}\times\frac{32}{103}=0.0194$.

\vspace{2mm}

Then we can get the NML code-length as $L=-log_2 p_{nml}(0011)=\textbf{5.6865}$

\end{document}